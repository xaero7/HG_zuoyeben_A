%\section{数据、信息与进位制}
\setcounter{section}{3}
\setcounter{subsection}{5}
\subsection{函数与模块}





\begin{groups}


\group{程序填空}{请在横线处将正确的代码补充完整,每空语句不会超过一句}
\begin{questions}[rp]

%% ============ 8
\setcounter{qnumber}{1}
\question ({\kaishu 作业本3.6})
某餐厅推出优惠活动,凡到店消费的顾客均可随机抽取三份小菜中的一份,消费20元以上再赠送一个“荷包蛋”。请用Python程序解决问题。 
\columnratio{0.6}
\begin{paracol}{2}
\begin{lstlisting}
import random
def coupon(money):
    `\clozeblank{2}`
    if 0 < money <= 20:
        return food
    elif money > 20:
        return food +", 荷包蛋"
appetizer = ["话梅花生", "拍黄瓜", "凉拌三丝"]
payment = float(input("输人您的消费金额:"))
`\clozeblank{2}`
print("赠送的小菜为:" + result)
\end{lstlisting}
\switchcolumn
\begin{lstlisting}[frame=single]
输入输出示例:

输入您的消费金额:25
赠送的小菜为:话梅花生, 荷包蛋
\end{lstlisting}
\end{paracol}



\begin{subquestions}
\subquestion 若输入的消费金额是30元,程序输出的可能结果有\key{~ 3 ~}种。
\subquestion 请将程序中划线处代码补充完整。
\end{subquestions}

\begin{solution}
\begin{lstlisting}
① food = random.choice(appetizer)
或 food = appetizer[random.randint(0, 2)]
② result = coupon(payment)
\end{lstlisting}
\end{solution}


%% ============ 9
\setcounter{qnumber}{1}
\question ({\kaishu 作业本3.6})
用Python编写程序,实现一元二次方程式$ax^2+bx+c=0 \, (a \ne 0)$的求解。输人系数$a,b,c$的值,输出方程可能的解。

\begin{subquestions}
\subquestion 编写函数$f()$实现一元二次方程根的判别式的求解,请完善程序。
\begin{lstlisting}
def f(x, y, z):
    a = y * y
    b = 4 * x * z
    `\clozeblank{2}`
\end{lstlisting}

\subquestion 请将主程序补充完整。
\begin{lstlisting}[numbers=left]
import math

a = int(input('系数a:'))
b = int(input('系数b:'))
c = int(input('系数c:'))
delta = f(a, b, c)
if `\clozeblank{2}`:
    x1 = (-b + sqrt(delta)) / (2 * a)
    x2 = (-b - sqrt(delta)) / (2 * a)
    print(x1, x2)
else:
    print("该方程无解")
\end{lstlisting}

\subquestion 小明在调试程序时发现错误提示“NameError:name 'sqrt' is not defined”,则下列修改可行的是({\kaishu 多选填字母,全部选对的得2 分,选对但不全的得1 分,不选或有选错的得0 分 })
\choice{第8、9行的sqrt改成math.sqrt,其他不改}
{第1行改成 import sqrt,其他不改}
{第1行改成 from math import sqrt,其他不改}
{第6行改成 delta=sqrt(a,b,c),其他不改}
\end{subquestions}

\begin{solution}
\begin{lstlisting}
① return a - b
② delta >= 0  或者 f(a, b, c) >= 0
AC
\end{lstlisting}
\end{solution}


%% ============ 10
\setcounter{qnumber}{1}
\question ({\kaishu 作业本3.6})
某次考试的满分是100分,50名学生的成绩都是整数得分,请以10分为等级统计各个等级人数。如8名学生成绩分别为[83, 89, 92, 100, 93, 95, 78, 98],则70$\sim$79有1人,80$\sim$89有2人,90$\sim$99有4人,100有1人。

\begin{subquestions}
\subquestion 假设列表$a$中保存了所有分数去除个位数的值,如83分存储为8,79分存储为7,100分存储为10……,函数check()统计了0$\sim$10各个数字出现的次数,并保存在列表$b$中,请完善程序。
\begin{lstlisting}
def check(a):
    b = [0] * 11
    for i in range(len(a)):
        `\clozeblank{2}`
    return b
\end{lstlisting}

\subquestion 主程序生成了50个随机数模拟学生成绩,并统计各个等级的人数,输出示例如右侧所示,请完善程序。
\columnratio{0.6}
\begin{paracol}{2}
\begin{lstlisting}
import random

lst = []
for i in range(50):
    num = random.randint(76, 100)
    # 在列表lst中添加一个元素
    lst.append(`\clozeblank{2}`)  
\end{lstlisting}
\switchcolumn
\begin{lstlisting}[frame=single]
输出示例:

70 ~ 79分人数有: 6
80 ~ 89分人数有: 28
90 ~ 99分人数有: 13
100分人数有: 3
\end{lstlisting}
\end{paracol}
\begin{lstlisting}
b = `\clozeblank{2}`
for i in range(len(b)):
    if b[i] > 0:
        print(i*10, "~", i*10+9, "分人数有:", b[i])
\end{lstlisting}

\end{subquestions}


\begin{solution}
\begin{lstlisting}
① b[a[i]] += 1
② num // 10
③ check(lst)
\end{lstlisting}
\end{solution}



%% ======================= 4
%% 24.04 高二温中双语学考模拟卷
\setcounter{qnumber}{1}
\question 若一个整数$n(n>10)$从左向右和从右向左读其结果相同,且是素数,则称$n$为回文素数,例如$133020331$是回文素数。下列Python 程序段用于找出$1000$以内的所
有回文素数。
\begin{subquestions}
\subquestion 函数hws()判断整数num是否是回文整数,若是则返回True,否则返回False,请完善程序。
\begin{lstlisting}
def hws(num):
    n = str(num)
    if `\clozeblank{2}`:
        return True
    return False 
\end{lstlisting}

\subquestion 函数sushu()判断整数num是否是素数,若$num$不能被$[2, \sqrt{num}]$内的任何一个整数整除,则$num$是素数,请完善程序。
\begin{lstlisting}
import math

def sushu(num):
    `\clozeblank{2}`
    for i in range(2, int(math.sqrt(num)) + 1):
        if `\clozeblank{2}`:
            flag = False
            break
    return flag  
\end{lstlisting}

\subquestion 请完善主程序,输出10 $\sim$ 1000的回文素数。
\begin{lstlisting}
n = 11
while n < 1000:
    if `\clozeblank{2}`:
        print(n)
    n = n + 2
\end{lstlisting}
\end{subquestions}


\begin{solution}
\begin{lstlisting}
① n == n[::-1]
② flag = True
③ num % i == 0
④ sushu(n) and hws(n)
\end{lstlisting}
\end{solution}




\end{questions}
\end{groups}
