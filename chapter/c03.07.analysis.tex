\setcounter{section}{3}
\setcounter{subsection}{6}
\subsection{解析算法的程序实现}

\begin{groups}

\group{程序填空}{请在横线处将正确的代码补充完整,每空语句不会超过一句}
\begin{questions}[rp]

%% ============ 1

\setcounter{qnumber}{1}
\question ({\kaishu 作业本3.7})
有如下数列:$1, 4, 5, 9, 14, 23, \cdots $ 该数列可以用以下递推的方式进行定义:
\[
F(n)=
\begin{cases}
1 & (n=1) \\
4 & (n=2) \\
\underline{\hspace{1cm}} & (n>2, n \in \mathbb{N})
\end{cases}
\]
则划线处的表达式为
\choice{$F(n+1)+F(n)$}{$F(n)+F(n-1)$}{$F(n-1)+F(n-2)$}{$F(n-1)+2$}

\begin{solution}
C
\end{solution}


%% ============ 2

\setcounter{qnumber}{1}
\question ({\kaishu 作业本3.7})
某超市每天傍晚 17:30 后对绿叶蔬菜类有买二送一活动:顾客购买任意三种质量均不超过 1000 克绿叶蔬菜(以 $a$, $b$, $c$ 表示),收费时会在总价 $s$ 中免收价格最低的一种蔬菜。则该活动的实际折扣(= 实付金额 / 总金额 $\times 100$,计算结果保留两位小数)的 Python 表达式为
\choice{\stt{int(s-min(a,b,c)/s*100)}}
{\stt{int((s-min(a,b,c))/s*100+0.5)}}
{\stt{round((s-max(a,b,c))/s*100,2)}}
{\stt{round((s-min(a,b,c))/s*100,2)}}

\begin{solution}
D
\end{solution}
 
%% ============ 3

\setcounter{qnumber}{1}
\question ({\kaishu 作业本3.7})
模拟一个简易计算器,其功能是:输入两个数和一个运算符(加、减、乘、除),进行算术运算并输出运算结果。

\begin{subquestions}
\subquestion 用 Python 语言编写的程序如下,请在划线处填入合适的语句或表达式,实现程序功能。
\begin{lstlisting}
a = float(input("输入第一个数:"))
ch = `\clozeblank{2}`
b = float(input("输入第二个数:"))
if ch == "+":
    print(a, ch, b, "=", a+b)
elif ch == "-":
    print(a, ch, b, "=", a-b)
elif ch == "*":
    print(a, ch, b, "=", a*b)
elif ch == "/":
    if `\clozeblank{2}`:
        print(a, ch, b, "=", a/b)
    else:
        print("除数不能为0")
else:
    print("运算符不正确")
\end{lstlisting}

\subquestion 当输入的第一个数为 4,运算符为“$\backslash$”,第二个数为 0,则程序输出的结果是
\choice{0}
{4}
{除数不能为 0}
{运算符不正确}
\end{subquestions}

\begin{solution}
\begin{lstlisting}
① input()
② b != 0
D
\end{lstlisting}
\end{solution}





\begin{solution}

\end{solution}


%% ============ 4
\setcounter{qnumber}{1}
\question ({\kaishu 作业本3.7})
某地块为四边形,地块边长分别为 $L1, L2, L3,L4$,如图所示。求该地块面积的 Python 程序如下,请在划线处填入合适代码。

\begin{minipage}{0.55\textwidth}
\begin{lstlisting}
# 利用海伦公式编写求三角形面积的函数area
def area(a, b, c):
    p = `\clozeblank{2}`
    s = sqrt((p * (p-a) * (p-b) * (p-c)))
    return s
# 求划分后的两个地块面积之和
def quad(a, b, c, d, e):
    s = `\clozeblank{2}`
    return s
\end{lstlisting}
\end{minipage}
\hfill
\begin{minipage}{0.4\textwidth}
\includegraphics[width=0.98\textwidth]{pic/c03.07.4area.jpg}
\end{minipage}

\begin{lstlisting}
# 主程序
from math import sqrt
f = []
for i in range(5):  # 依次将图中划分后的五条边的值添加到列表f中
    a = float(input("输入一条边的值:"))
    f.append(a)     # 在列表f中添加一个元素a
s = quad(f[0],f[1],f[2],f[3],f[4])
print(s)
\end{lstlisting}

\begin{solution}
\begin{lstlisting}
① (a + b + c) / 2
② area(a, b, e) + area(c, d, e)
\end{lstlisting}
\end{solution}


%% ============ 5
% 作业本第5题跳过


%% ============ 6
\setcounter{qnumber}{1}
\question ({\kaishu 作业本3.7})
编写一个摄氏温度转换成华氏温度的 Python 程序,实现功能:输入两个摄氏温度数据(整数),输出包含两列数据,第 1 列为两数之间的摄氏温度值(间隔 1 度、升序),第 2 列为对应的华氏温度值。其中,摄氏温度 $x$ 转换成华氏温度 $y$ 的计算公式为:$y = \frac{9}{5}x + 32$。

输入、输出格式分别如左、右侧所示:

\begin{minipage}{0.5\textwidth}
\begin{lstlisting}
输入第1个数:
15
输入第2个数:
10  
\end{lstlisting}
\end{minipage}
\hfill 
\begin{minipage}{0.5\textwidth}
\begin{lstlisting}
摄氏温度      华氏温度
10            50.0
11            51.8
12            53.6
13            55.4
14            57.2
15            59.0
\end{lstlisting}
\end{minipage}

实现上述功能的某 Python 程序如下,在方框处补充一段程序代码,完善程序,实现程序功能。
\begin{lstlisting}
t1 = int(input("输入第1个整数:\n"))
t2 = int(input("输入第2个整数:\n"))
if t1 > t2:
    t1, t2 = t2, t1
print("摄氏温度" + "   " + "华氏温度")
i = t1
while i <= t2:
    print(i, `\clozeblank{2}`)
    `\clozeblank{2}`
\end{lstlisting}

\begin{solution}
\begin{lstlisting}
9 * i / 5 + 32
i = i + 1
\end{lstlisting}
\end{solution}


%% ============ 7
\setcounter{qnumber}{1}
\question ({\kaishu 作业本3.7})
测量高度问题。如图所示,$M$是山体的最高点,其底部 $N$ 点不可到达。如何测量该山体的高度 $h$?
\vspace*{-5mm}
\begin{figure}[!h]
\centering
\includegraphics[width=0.4\linewidth]{pic/c03.07.7mountain.jpg}
\end{figure}
\vspace*{-5mm}
\begin{subquestions}
\subquestion 若$B$ 点到 $N $点的长度为 $bn$,$A$ 点到 $N$ 点的长度为 $an$,$MN$ 的高度为 $h$,$A$ 点到 $B$ 点的长度为 $ab$,点 $A$ 对点 $M$ 的仰角为 $a$,点 $B$ 的仰角为 $b$。通过 math 模块中的正切函数 $\tan(x)$ 公式可得:
\begin{lstlisting}
an = h / math.tan(a * math.pi / 180)
bn = `\clozeblank{2}`
\end{lstlisting}

由于 $ab$ 的长度可以通过测量得到,且 $AB$ 的长度 $ab = bn - an$。将上述两个表达式代入,可得:
\stt{h = ab/(1/math.tan(b * math.pi / 180) - 1/math.tan(a * math.pi / 180))}

\subquestion 解决该问题的 Python 程序如下,请在下述程序划线处填入合适的代码。

\end{subquestions}

\begin{lstlisting}
import math
ab = float(input()); a = float(input()); b = float(input())
cota = 1 / math.tan(a * math.pi / 180)
cotb = 1 / math.tan(b * math.pi / 180)
h = ab / (`\clozeblank{2}`)
print(round(h, 2))
\end{lstlisting}

\begin{solution}
\begin{lstlisting}
h / math.tan(b * math.pi / 180)
cotb - cota
\end{lstlisting}
\end{solution}

\end{questions}
\end{groups}