%\section{数据、信息与进位制}
\setcounter{section}{4}
\setcounter{subsection}{2}
\subsection{利用pandas模块处理数据}



\begin{groups}


\group{程序填空}{请在横线处将正确的代码补充完整,每空语句不会超过一句}
\begin{questions}[rp]

%% ============ 6
\setcounter{qnumber}{1}
\question ({\kaishu 作业本4.3})
2018年,我国流通领域重要生产资料市场价格数据的示例如图所示。编写Python程序,统计每个产品的最高价格和平均价格。
\begin{figure}[!h]
\centering
\includegraphics[width=0.6\linewidth]{pic/c04.03.6data}
\end{figure}

\columnratio{0.5}
\begin{paracol}{2}
\begin{lstlisting}
import pandas as pd
df = pd.read_csv("data.csv")
df_g = `\fbox{\color{white}I\hspace{3cm}}`
df_max = df_g["价格(元)"].max()
df_mean = `\clozeblank{2}`
print(df_max)
print(df mean)
\end{lstlisting}
\switchcolumn
\begin{lstlisting}[frame=single]
加框处可选语句:
(A). df.groupy("类别")
(B). df.groupy("产品名称")
(C). df.groupy("价格(元)")
\end{lstlisting}
\end{paracol}
实现该功能的Python程序如上所示,加框处应选:\key{~ B ~},并将划线处补充完整。

\begin{solution}
B;
df\_g["价格(元)"].max()
\end{solution}


%% ============ 7
\setcounter{qnumber}{1}
\question ({\kaishu 作业本4.3})
某同学收集了部分城市2022年4月每天24小时空气主要污染物浓度数据,按天分别保存在CSV文件中,如4月1日的数据保存在“20220401.csv”中,数据格式示例如图所示:
\begin{figure}[!h]
\centering
\includegraphics[width=0.7\linewidth]{pic/c04.03.7AQI}
\end{figure}
\begin{subquestions}
\subquestion 定义函数dave,功能是读取数据文件,计算各主要空气污染物的日平均浓度,并返回AQI的日平均浓度,代码如下:
\begin{lstlisting}
def dave(fn):
    df = pd.read_csv(fn)
    df1 = df[df.columns[2:]]  # 取type及其后的所有列
    g = dfl.groupby(`\clozeblank{2}`, as_index=False).mean()
    return `\clozeblank{2}`
\end{lstlisting}
请在划线①处应填入合适的代码,并为划线②处选填合适的代码({\kaishu 单选,填字母})
\choice{\texttt{df[df.type=='AQI']}}
{\texttt{df1.AQI}}
{\texttt{g.AQI}}
{\texttt{g[g.type=='AQI']}}

\subquestion 统计2022年4月1日十个城市中空气质量优或良的城市个数(空气污染指数AQI$\le 50$为优,AQI$\le 100$为良),代码如下,请将划线处程序补充完整:
\begin{lstlisting}
filename = "20220401.csv"
aqi = `\clozeblank{2}`
n = 10;  ans = 0
city = aqi.columns[1:n+1]  # columns可以取得数据框对象aqi的所有列名称
for i in range(n):
	if aqi.at[`\clozeblank{2}`] <= 100:
	    ans += 1
print(ans)
\end{lstlisting}
\end{subquestions}

\begin{solution}
\begin{lstlisting}
"type"
D
dave(filename)
0, city[i]
\end{lstlisting}
\end{solution}


%% ============ 例2
\setcounter{qnumber}{1}
\question ({\kaishu 作业本4.3})
小王从网站上收集了2020年浙江省各区县气象台发布的预警数据,数据集格式如图所示。为了分析各气象台的预警情况,小王编写了如下Python程序
\begin{figure}[!h]
\centering
\includegraphics[width=0.6\linewidth]{pic/c04.03.weather}
\end{figure}
\begin{subquestions}
\subquestion 若要显示2020年暴雨预警次数超过15次的气象台信息,请在划线处给出\udt{两种}实现方式
\begin{lstlisting}
import pandas as pd
import matplotlib.pyplot as plt
df = pd.read_csv("data_weather.csv")
print(`\clozeblank{2}`)
\end{lstlisting}

\subquestion 若要了解该年中没有出现过的极端天气的情况(如图示中的“暴雪”),并把这些天气预警数据列从数据表中去除,请完善下面的代码。
\begin{lstlisting}
for i in df.columns[2:]:         # columns可以取得数据框df的所有列名称
    if `\clozeblank{2}`:
        df = df.drop(i, axis=1)  # 根据列名称$i$删除列
\end{lstlisting}

\subquestion 小王想做一份浙江省各地市台风天气分析汇报,针对各地市发生台风次数进行比较分析,添加的程序段与绘图结果如下:
\columnratio{0.5}
\begin{paracol}{2}
\begin{lstlisting}
df2 = df.groupy("市区").sum()
plt.figure(figsize=(10,5))
`\clozeblank{2}`
plt.titile("2020年各地市台风预警情况")
plt.xlabel("地市")
plt.ylabel("台风预警次数")
plt.show()
\end{lstlisting}
\switchcolumn
\includegraphics[width=\linewidth]{pic/c04.03.weather2}
\end{paracol}

划线处的代码应为
\choice{\texttt{plt.bar(df2.\!市区, df2.\!台风)}}
{\texttt{plt.bar(df2.index, df2.\!台风)}}
{\texttt{plt.plot(df2.columns[0:], df2.\!台风)}}
{\texttt{plt.plot(df2.index, df2.台风)}}
\end{subquestions}

\begin{solution}
\begin{lstlisting}
① df[df["暴雨"] > 15]  或 df[df.暴雨 > 15] 
② df[i].sum() == 0
③ B
\end{lstlisting}
\end{solution}




\end{questions}
\end{groups}
