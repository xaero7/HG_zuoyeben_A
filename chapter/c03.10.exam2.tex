\setcounter{section}{3}
\setcounter{subsection}{9}
\subsection{第三章自测试题}

\begin{groups}

\group{选择题}{单项选择}
\begin{questions}[rp]

%% =========== 1
\question
下列表达式的值为\stt{False}的是
\choice{\stt{8 \% 3 == 6 // 3}}
{\stt{not(8 + 3 < 8 - 3)}}
{\stt{2 ** 3  + 1 == 7}}
{\stt{10 / 3 != 5}}
\begin{solution}
C。表达式`8%3=2`,`6//3=2`,故选项A为True。`not(8+3<8-3)`的值为True。选项C左边等于9,右边为7,故返回值为False。选项D的值也为True。
\end{solution}

%% =========== 2
\question
已知字典\stt{s = \{1: 2, 'a': 'b', 2: 3, 'b': 4\}},下列说法正确的是( )
\choice{\stt{s[1] + s['b']}的值为24}
{\stt{s[1] + s['a']}的值为\stt{'2b'}}
{\stt{4 in s}的值为\stt{False}}
{\stt{s[2] in s}的值为\stt{True}}
\begin{solution}
C。暂无解析
\end{solution}

%% =========== 3
\question
下列 Python 表达式的值为数值$5$的是
\choice{\stt{int(max(["36","100","5"]))}}
{\stt{len([1,2,3,"IT"])}}
{\stt{ord("e")-ord("a")}}
{\stt{str(abs(-5))}}
\begin{solution}
A。暂无解析
\end{solution}

%% =========== 4
\question
\smkai{(全A+ )}下列关于Python表达式描述正确的是
\choice{\stt{5 ** 3 // 2 + 4}的值为9}
{\stt{13 \% 9 // 4 / 2 >= 1}的值为True}
{\stt{"x" not in "cx" and 3 + 4 > 5}的值为True}
{\stt{int(6 // 4 / 2 + 0.5)}的值为1}
\begin{solution}
D。D本题考查 Python 基本运算符。选项A,结果为66;
选项B,结果0.5>=1 返回结果为False;
选项C,and 运算一假则假, 'x' not in 'cx'返回False。
故选D。

\end{solution}

%% =========== 5
\question
某店铺优惠活动:消费满688后打7折,不满额则9折。计算优惠价格的Python 程序段如下:
\begin{lstlisting}
if x >= 688:  
    y = x * 0.7  
else:  
    y = x * 0.9
\end{lstlisting}
下列选项中与上述功能相同的是:
\choice{程序A}
{程序B}
{程序C}
{程序D}\\
\begin{minipage}{0.246\textwidth}
程序A
\begin{lstlisting}
if x >= 688:
    y = x * 0.7
    y = x * 0.9
\end{lstlisting}
\end{minipage}
\begin{minipage}{0.246\textwidth}
程序B
\begin{lstlisting}
y = x * 0.7
if x >= 688:
    y = x * 0.9
\end{lstlisting}
\end{minipage}
\begin{minipage}{0.246\textwidth}
程序C
\begin{lstlisting}
if x >= 688:
    y = x * 0.7
else:
y = x * 0.9
\end{lstlisting}
\end{minipage}
\begin{minipage}{0.246\textwidth}
选项D
\begin{lstlisting}
if x >= 688:
    y = x * 0.7
if x < 688
    y = x * 0.9
\end{lstlisting}
\end{minipage}

\begin{solution}
D。暂无解析
\end{solution}

%% =========== 6
\question
有如下Python程序段:
\begin{lstlisting}
s = 0
for i in range(0, 7):
    if i % 3 == 0:
        s = s + 1
    if i % 3 == 1:
        s = s + 1
    else:
        s = s + 2
print(s)
\end{lstlisting}
其输出结果为
\choice{14}
{15}
{16}
{17}
\begin{solution}
B。注意并列的两个if
\end{solution}

%% =========== 7
\question
某 Python 程序如下:
\begin{lstlisting}
s = "HangZhou-0571"
y = ""
for i in range(len(s)):
    c = s[i]
    if c >= "A" and c <= "Z":
        c = chr(ord(c) - 1)
    elif c >= "0" and c <= "9":
        c = chr(ord(c) + 1)
    y = c + y
print(y)
\end{lstlisting}
程序运行后,输出的结果是:
\choice{GangYhou-1682}
{2861-uohYgnaG}
{gangyhou-1682}
{2861-uohygnag}
\begin{solution}
B。无
\end{solution}

%% =========== 8
\question
以下Python程序段执行后,结果为2,
\begin{lstlisting}
def f(list):
    m = list[0]
    for i in range(1, len(list)):
        if _______________:
            m = list[i]
    return m
s = [4, 5, 2, 6, 5, 8]
print(f(s))
\end{lstlisting}
那么横线处的代码为
\choice{m > list[i]}
{m < list[i]}
{m == list[i]}
{m != list[i]}
\begin{solution}
A。无
\end{solution}

%% =========== 9
\question
下列关于算法的说法,正确的是
\choice{“输出所有的素数”问题,可以用枚举算法解决}
{在屏幕上输出“Hello World!”,这个任务不需要用算法解决}
{“鸡兔同笼”问题,用解析算法和枚举算法都可以解决}
{算法必须要有输入,如“鸡兔同笼”问题必须输入头和脚的数量}
\begin{solution}
C。本题考查算法的基本特征。算法必须是有穷的,而数学上已经证明,素数有无穷多个﹐A错误。算法就是解决问题的一系列步骤﹐输出“Hello World !”,也是一个问题,B错误。算法可以没有输入,D错误。

\end{solution}

%% =========== 10
\question
有如下Python程序段:
\begin{lstlisting}
ma = a[0]; mb = a[0]
pa = pb = 0
for i in range(1, 10):
    if ma < a[i]:
        ma = a[i]; pa = i
    if mb > a[i]:
        mb = a[i]; pb = i
\end{lstlisting}
下列关于该程序段功能描述正确的是
\choice{变量pa存储a[0]至a[9]中的最大值}
{变量ma存储a[0]至a[9]中的最小值}
{变量pb存储a[0]至a[9]中的最大值}
{变量mb存储a[0]至a[9]中的最小值}
\begin{solution}
D。D本题考查枚举算法中最值的查找,以及数组索引。从遍历中条件语句可知, ma保存最大值, pa是最大值索引, mb保存最小值,pb 是最小值索引。故选 D。

\end{solution}


\end{questions}
\end{groups}