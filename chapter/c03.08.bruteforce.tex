\setcounter{section}{3}
\setcounter{subsection}{7}
\subsection{枚举算法的程序实现}

\begin{groups}

\group{程序填空}{请在横线处将正确的代码补充完整,每空语句不会超过一句}
\begin{questions}[rp]

%% ============ 1
\question 
下列问题中,适合使用枚举算法的是
\choice{计算两个电阻的并联值}{计算五位同学的平均身高}{查找 100 以内所有能被 6 整除的数}{计算超市的月利润}

\begin{solution}
C
\end{solution}

%% ============ 2
\question 
有如下 Python 程序段:
\begin{lstlisting}
s = 0
for i in range(1,101):
    if i % 2 == 0:
        s += i
\end{lstlisting}
该程序段被执行后,变量 $s$ 中存储的是 1~100 之间的
\choice{偶数个数}{奇数个数}{奇数之和}{偶数之和}

\begin{solution}
D
\end{solution}

%% ============ 3
\setcounter{qnumber}{1}
\question 
求方程组的正整数解。有如下方程组:
\[
\begin{cases}
2x^3 + xy - y = 302 \\
5x + y = 38
\end{cases}
\]
求该方程组正整数解的 Python 程序如下,请将程序划线处的代码补充完整。
\begin{lstlisting}
for x in range(1, 8, 1):
    if `\clozeblank{2}`:
        print("x=", x)
        print("y=", 38 - 5 * x)
\end{lstlisting}

上述程序中,用到的主要算法是 \underline{\hspace{3cm}}。

\begin{solution}
\begin{lstlisting}
(1) 枚举算法
(2) 2*x**3 + x*(38-5*x) - (38-5*x) == 302
\end{lstlisting}
\end{solution}

%% ============ 4
\setcounter{qnumber}{1}
\question 
在字符串 \stt{stra} 中删除字符串 \stt{strb}。例如,\stt{stra} 是 "\stt{aabbcccddcccee}",\stt{strb} 是 "\stt{ccc}",则得到的新字符串是 "\stt{aabbddee}"。解决该问题的 Python 程序如下,请在程序的划线处填入合适的代码。

\begin{lstlisting}
stra = input("请输入字符串:")
strb = input("请输入要删除的字符串:")
t = "";      i = 0
while i < len(stra)-len(strb)+1:
    if stra[i:`\clozeblank{2}`] != strb:
        t += stra[i]
        i += 1
    else:
        i += len(strb)
`\clozeblank{2}`
print(t)
\end{lstlisting}

\begin{solution}
\begin{lstlisting}
① i+len(strb)
② t += stra[i:]
\end{lstlisting}
\end{solution}

%% ============ 5
\setcounter{qnumber}{1}
\question 
输入数字字符串,找出其中的最大值。这是常见的字符处理问题。如输入字符串 "24,34,93,102,2,4,1",找到的数字最大值为 102。编写 Python 程序实现这一功能的程序如下:
\begin{lstlisting}[numbers=left]
str1 = input("请输入数字串:")
i = 0;     maxnum = 0
for j in range(len(str1)):
    if str1[j] > '9' or str1[j] < '0':
        s = int(str1[i:j])
        if s > maxnum:
            maxnum = s
        i = j + 1
s = int(str1[i:])
if s > maxnum:
    maxnum = s
print("最大值为:", maxnum)
\end{lstlisting}

若删除第9~11行代码,程序可能无法正确找出最大值。能测试出这个问题的数据是
\choice{"24,34,93,102,2"}{"24,34,93,2,102"}{"93,24,102,2,34"}{"24,34,102,2,93"}

\begin{solution}
B
\end{solution}


%% ============ 6
\setcounter{qnumber}{1}
\question 
某网络服务平台要求新注册用户的密码必须以字母开头,并且含有字母(区分大小写)、数字和下划线,密码字符的长度为 6~18 个字符。判断用户密码是否合法的程序如下,请完善程序。

\begin{lstlisting}
n1 = False            # 判断数字
n2 = False            # 判断下划线
flag = False          # 判断首字符和长度是否满足要求
password = input('请输入密码:\n')
n = `\clozeblank{2}`
ch = password[0]
if 19 > n > 5 and ('z' >= ch >= 'a' or 'Z' >= ch >= 'A'):
    flag = True
if flag:
    for i in range(1, n):
        ch = password[i]
        if '9' >= ch >= '0':
            n1 = `\clozeblank{2}`
        elif ch == '_':
            n2 = True
if `\clozeblank{2}`:
    print('结果: 合法')
else:
    print('结果: 不合法')
\end{lstlisting}

\begin{solution}
\begin{lstlisting}
① len(password)
② True
③ n1 and n2 或者填写 n1 and n2 and flag
\end{lstlisting}
\end{solution}



\setcounter{qnumber}{1}
\question 
小明编写了Python程序来读取文本文件 \texttt{工资短信.txt}中各项收入(均为整数),并求得月工资总额。程序的运行结果如下,请在划线处填入合适的代码,完善程序。
\begin{lstlisting}[frame=single]
“工资短信.txt”文件内容:岗位工资3420,薪级工资3337,工龄补贴420,山区补贴300,
工作补贴480,月考勤奖400,交通津贴600,岗位津贴680。
程序输出结果:月工资总额:9637元
\end{lstlisting}
\begin{lstlisting}
f = open('工资短信.txt','r')
mx = f.read()
s = 0;  t = 0
for i in mx:
    if `\clozeblank{2}`:
        t = t * 10 + int(i)
    else:
        `\clozeblank{2}`
        t = 0
print("月工资总额:", str(s), "元")
\end{lstlisting}

\begin{solution}
\begin{lstlisting}
① i >= '0' and i <= '9'
② s += t
\end{lstlisting}
\end{solution}






%% ============ 5
\setcounter{qnumber}{1}
\question 
素数(prime number)又称质数,在大于 1 的自然数中,除了 1 和它本身以外不再有其他因数的数称为素数,如 2, 3, 5, 7, 11, …… 求 1~100 之间所有素数的 Python 程序如下,请完善程序。

\begin{lstlisting}
import math
list1 = [2, 3]
n = 0
for i in range(5, 101, 2):
    for j in range(2, `\clozeblank{2}`):
        if `\clozeblank{2}`:
            break
        else:
            list1.append(i)  # 将素数添加到列表list1中
        n += 1
for i in range(0, n+2):
    print(list1[i])
\end{lstlisting}

\begin{solution}
\begin{lstlisting}
① i 或 int(math.sqrt(i))+1 或其他等价表达式
② i%j == 0 或其他等价表达式
\end{lstlisting}
\end{solution}






\end{questions}
\end{groups}