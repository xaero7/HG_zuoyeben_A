\setcounter{section}{3}
\setcounter{subsection}{9}
\subsection{单元练习}

\begin{groups}

\group{选择题}{在每小题给出的四个选项中,只有一项是符合题目要求的}
\begin{questions}[rp]

%% ============ 1
\question
有Python程序段如下:
\begin{lstlisting}
a = "123";   b = 123;    c = `\underline{\hspace*{2cm}}`
\end{lstlisting}
若变量$c$的值为整型数据123123,则程序划线处的正确表达式是
\choice{a * 2}
{a + a}
{int(a) * 1000 + a}
{int(a + a)}

\begin{solution}
D
\end{solution}


%% ============ 2
\question
下列Python表达式的值为5的是
\choice{\stt{len(range(0,5))}}
{\stt{max(5, 10, 15)}}
{\stt{len([1, 5])}}
{\stt{ord('5')}}

\begin{solution}
A
\end{solution}


%% ============ 3
\setcounter{qnumber}{3}
\question
有如下Python程序:
\begin{lstlisting}
s = "Happy New Year!";   m = `\underline{\hspace*{2cm}}`
if m == "Happy":
    print("Happy to you!")
elif m == "Year":
    print("Good Luck!")
else:
    print("It's a fine day!")
\end{lstlisting}
运行该程序,若得到的结果是"Good Luck!",则划线处代码是
\choice{s[9:13]}
{s[10:13]}
{s[10:14]}
{s[11:14]}

\begin{solution}
C
\end{solution}


%% ============ 4
\question
成年人的静息心率通常为60~100次/分,低于60称为心动过缓,高于100称为心动过速。现将测得的某人心率存入变量\stt{rate},下列Python程序段\udt{不能}输出心率状态的是
\choice{选项A}{选项B}{选项C}{选项D}

\begin{minipage}{0.48\textwidth}
\begin{lstlisting}
选项A.  sta = "心动过缓"
        if rate > 100:
            sta = "心动过速"
        elif rate <= 100:
            sta = "正常"
        print(sta)
\end{lstlisting}
\end{minipage}
\begin{minipage}{0.48\textwidth}
\begin{lstlisting}
选项B.  if rate > 100:
            sta = "心动过速"
        elif rate < 60:
            sta = "心动过缓"
        else:
            sta = "正常"
        print(sta)
\end{lstlisting}
\end{minipage}
\begin{minipage}{0.48\textwidth}
\begin{lstlisting}
选项C.  sta = "心动过速"
        if rate < 60:
            sta = "心动过缓"
        elif rate <= 100:
            sta = "正常"
        print(sta)
\end{lstlisting}
\end{minipage}
\begin{minipage}{0.48\textwidth}

\begin{lstlisting}
选项D. if rate < 60:
          sta = "心动过缓"
       if rate > 100:
           sta = "心动过速"
       if rate >= 60 and rate <= 100:
           sta = "正常"
       print(sta)
\end{lstlisting}
\end{minipage}

\begin{solution}
A
\end{solution}

%% ============ 5
\setcounter{qnumber}{1}
\question
某算法的部分程序代码及其相对应的流程图如下:

\begin{minipage}{0.55\linewidth}
\begin{lstlisting}
s = 1
for i in range(1, `\clozeblank{1}`):
    if i % 2 != 0:
        `\clozeblank{2}`
print(s)
\end{lstlisting}
\end{minipage}
\hfill
\begin{minipage}{0.4\linewidth}
\centering
\includegraphics[width=0.8\linewidth]{pic/c03.09.5flow.jpg}
\end{minipage}

则程序中划线①②处应填入的是
\choice{\stt{①11 ②s*=i}}
{\stt{①11  ②s*=i+1}}
{\stt{①10  ②s*=i}}
{\stt{①10  ②s*=i-1}}

\begin{solution}
A
\end{solution}


%% ============ 6
\setcounter{qnumber}{6}
\question
若$x$是实型变量,则下列选项中,与表达式 \stt{ x > 0 and not x > 5} 等价的是
\stt{①not(x <= 0 and x > 5)} \stt{② not(x <= 0 or x > 5) ③ x > 0 and x <= 5 ④ x > 0 or x <= 5}
\choice{①③}{①④}{②③}{②④}

\begin{solution}
C
\end{solution}


%% ============ 7
\question
有如下Python程序段:
\begin{lstlisting}
s = "6st-udYy"
t = ""
for i in range(len(s)):
    if s[i] >= "a" and s[i] <= "z":
        t = t + s[i]
print(t)
\end{lstlisting}
该程序段的功能是输出字符串$s$中的
\choice{小写字母个数}{所有小写字母}{所有数字之和}{所有非数字字符}

\begin{solution}
B
\end{solution}


%% ============ 8
\question
斐波那契数列又称黄金分割数列,指的是这样一个数列:$0, 1, 1, 2, 3, 5, 8, 13, 21, 34, \cdots$ 在数学上,斐波那契数列以下面的方法来定义:
$ F(0) = 0 \ (n=0), F(1) = 1 \ (n=1), F(n) = F(n-1) + F(n-2) \ (n \geq 2)$,输出斐波那契数列中第$10$个数的Python程序如下:
\begin{lstlisting}
def fib(n):
    a, b = 0, 1
    for i in range(n):
        `\fbox{\stt{a, b = b, a + b}}`
    return a
print(fib(9))
\end{lstlisting}
关于加框处的代码,下列说法正确的是
\choice{最多执行1次}{执行次数可以无限次}{交换a和b的值}{从数列中的第三项起,每一项都是它相邻的前两项之和}

\begin{solution}
D
\end{solution}


%% ============ 9
\question
有如下Python程序:
\begin{lstlisting}
str1 = input("请输入一个二进制整数:")
str1_len = len(str1)
s = 0
for i in range(str1_len):
    a = int(str1[i])
    b = 2 ** (str1_len - i - 1)
    x = a * b
    s += x
print(s)
\end{lstlisting}
运行该程序,输入1101110,程序结果为
\choice{110}{111}{220}{1101110}

\begin{solution}
A
\end{solution}


%% ============ 10
\question
有10个数据$34,22,101,8,14,88,24,17,54,7$依次存放在列表list1中,有如下Python程序:
\begin{lstlisting}
list1 = [34, 22, 101, 8, 14, 88, 24, 17, 54, 7]
num = list1[0]
for i in range(1, 10):
    if list1[i] < num:
        num = list1[i]
print(num)
\end{lstlisting}
当程序运行结束时,输出的值是
\choice{101}{7}{8}{88}

\begin{solution}
B
\end{solution}


%% ============ 11
\setcounter{qnumber}{1}
\question
字符串生成程序。程序功能:输入由多个正整数组成的字符串$t1$(这些正整数以“,”为分割符和结束符),并以这些整数为位置信息,依次从字典$t2$中提取相应字符并连接成字符串$s$,最后将字符串$s$输出。其中,字典内容$t2$从文件“dictionary.txt”中读取。

如:\stt{t1 = "2,8,15,"},且$t2$从文件“dictionary.txt”中读取的内容为“python is a programming language.” 程序最终输出的字符串$s$为:\stt{yio}。

实现上述功能的Python程序如下,请完善程序。
\begin{lstlisting}
f = open('dictionary.txt', 'r')
t1 = input('字符位置:')
t2 = f.read()  # 从文件中读取字典内容
s = ""
t = ""
for i in range(`\clozeblank{2}`):
    c = t1[i]
    if c == ',':
        p = int(t)
        s = s + t2[p-1]
        t = ""
    else:
        `\clozeblank{2}`
print('生成内容:', s)
f.close()  # 关闭文件
\end{lstlisting}

\begin{solution}
① len(t1) \\
② t = t + c
\end{solution}


%% ============ 12
\setcounter{qnumber}{1}
\question
在平面直角坐标系中,给定一组有序的点。从原点出发,依次用线段连接这些点,构成一条折线,要求编写一个“计算折线长度”的Python程序,该Python程序代码编写思路如下:

{\kaishu
① 输入各点的坐标(最后一个点的坐标后不加逗号),存入变量 $a$ 中;\\
② 将原点坐标加到坐标序列 $a$ 的最前端; \\
③ 以逗号为界从 $a$ 中取出各点的坐标,存入列表 $b$ 中;其中列表 $b$ 中的第一个和第二个数字为第一个点的 $x$ 坐标和 $y$ 坐标,第三个和第四个数字为第二个点的 $x$ 坐标和 $y$ 坐标,以此类推;  
④ 计算折线长度。若以 $b[i]$ 标记为相邻两个点中前者的 $x$ 坐标,则 $b[i+2]$ 为后者的 $x$ 坐标,$b[i+1]$ 为前者的 $y$ 坐标,$b[i+3]$ 为后者的 $y$ 坐标。i的最小值为0,最大值为$len(b)-4$,步长为2。运用for句求折线长度。
}
\begin{lstlisting}
from math import sqrt
a = input("请输入各点的坐标:")
`\clozeblank{2}`     # 补上原点坐标
b = a.split(",")     # 以逗号为界将坐标分割后存入列表b中
if len(b) % 2 == 1:
    print("输入的坐标有误")
else:
    s = 0
    for i in range(0, len(b) - 2, `\clozeblank{1}`):
        s = s + sqrt((int(b[i+2])-int(b[i]))**2 + (int(b[i+3])-int(b[i+1]))**2)
    else:
        print("折线长度为:", s)
\end{lstlisting}

\begin{subquestions}
\subquestion 若三个点的坐标为$(5,10),(8,12),(6,17)$,则输入数据应为 \underline{\hspace{4cm}}。
\subquestion 请在划线处填入合适的代码。
\end{subquestions}

\begin{solution}
(1) 5,10,8,12,6,17 \\
(2) a = "0,0," + a \\
(3) 2
\end{solution}


%% ============ 13
\setcounter{qnumber}{13}
\question
小明设计了一个加密程序,对于任意输入的一个字符串,运行程序后会输出相应的密文。其中,字符加密算法的思路如下:

{\kaishu
(1) 只针对英文字母和数字进行加密,其余字符保持不变;  \\
(2) 将原文中的小写字母转换成大写字母;  \\
(3) 根据第2步所得结果,若是英文字母,则逐个后移4位(例如“A”→“E”,“Z”→“D”);若是数字,则逐个前移2位(例如:“3”→“1”,“1”→“9”)。
}

实现上述要求的某Python程序如下,但程序加框处代码有误,请修改。
\begin{lstlisting}
s1 = input("请输入明文:")
s2 = ""
for i in range(len(s1)):
    s = ""
    if s1[i] >= "a" and s1[i] <= "z":
        s = chr(ord(s1[i]) - 32)
        s = chr((ord(s) - ord("A") + 4) % 26 + ord("A"))
    elif s1[i] >= "A" and s1[i] <= "Z":
        s = chr((ord(s1[i]) - ord("A") + 4) % 26 + ord("A"))
    elif s1[i] >= "0" and s1[i] <= "9":
        `\fbox{s = chr((ord(s1[i]) - ord("A") - 2) % 10 + ord("A"))}`
    else:
        `\fbox{s2 = s2 + s1}`
    s2 = s2 + s
print("得到的密文是:", s2)
\end{lstlisting}

\begin{solution}
① chr((ord(s1[i]) - ord('0') - 2) \% 10 + ord('0')) 或者 s = str(  (int(s1[i]) + 8) \% 10  )\\
② s2 = s2 + s1[i]
\end{solution}


%% ============ 14
\setcounter{qnumber}{14}
\question
小范想要编写一个Python程序,实现统计离散数据中的“极大值”,即统计数值大于左右相邻两数的数。假设数据序列的左右端点不计入统计极大值的个数,且相邻相等的两数也不计入统计。例如:若输入数据 \stt{2 3 1 4 6 2}(以空格分隔),则极大值为 \stt{3,6}。

\begin{subquestions}
\subquestion 若输入以下10个数据:\stt{5 78 67 8 64 78 98 4 5 7},则极大值为 \underline{\hspace{4cm}}。
\subquestion 若极大值存储在列表max\_list中,请将代码填写完整。
\end{subquestions}

\begin{lstlisting}
num = input("输入数据:").split()     # 将以空格分隔的数字字符串存储在列表num中
numlist = list(map(int, num))       # 将列表各元素转换成整型存储在列表numlist中
i = 1
max_list = []
while i < len(numlist) - 2:
    if `\clozeblank{2}`:
        max_list.append(numlist[i])
    i += 1
print(max_list)
\end{lstlisting}

\begin{solution}
(1) 78, 98 \\
(2) numlist[i] > numlist[i-1] and numlist[i] > numlist[i+1]
\end{solution}

\end{questions}
\end{groups}