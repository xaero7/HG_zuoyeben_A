%\section{数据、信息与进位制}
\setcounter{section}{4}
\setcounter{subsection}{3}
\subsection{数据可视化}



\begin{groups}


\group{程序填空}{请在横线处将正确的代码补充完整,每空语句不会超过一句}
\begin{questions}[rp]

%% ============ 5
\setcounter{qnumber}{1}
\question ({\kaishu 作业本4.4})
使用Python编程研究一组幂函数的图像问题,程序如下:
\columnratio{0.5}
\begin{paracol}{2}
\begin{lstlisting}
import matplotlib.pyplot as plt
import numpy as np
# 生成-1到1之间包含50个数的等差数列
x = np.linspace(-1, 1, 50)
plt.plot(x, x ** 0.3, label="y=x**0.3")
plt.plot(x, x ** 0.5, label="y=x**0.5", linewidth=4)
plt.plot(x, x, label="y=x")
`\clozeblank{2}`
plt.plot(x, x ** 3, label="y=x**3", linewidth=3)
plt.legend()
plt.show()
\end{lstlisting}
\switchcolumn
\includegraphics[width=\linewidth]{pic/c04.04.5function}
\end{paracol}
对照图像的输出结果阅读上述Python程序。模仿其他行程序,用\udt{散点图}的形式绘制$y=x^2$的函数图像,请在划线处将程序补充完整。

\begin{solution}
plt.scatter(x, x ** 2, label="y=x**2")
\end{solution}


%% ============ 6
\setcounter{qnumber}{1}
\question ({\kaishu 作业本4.4})
采集某市某辆出租车2007年2月20日全天行驶轨迹的数据集,如图所示。该数据集特征包括:出租车ID、时间、经度、纬度、夹角角度、出租车的瞬时速度和出租车载客状态。
\begin{figure}[!h]
\centering
\includegraphics[width=0.7\linewidth]{pic/c04.04.6taxi}
\end{figure}
\begin{subquestions}
\subquestion 图所示数据集文件名为\key{~Taxi\_105.txt~},出租车ID、时间、经度、纬度、夹角角度、出租车的瞬时速度和出租车载客状态数据的间隔符为\key{~逗号t~},经度、纬度的数据类型为\key{~float~}({\kaishu 选填:float / int / str})。

\subquestion 绘制该数据集中出租车行驶轨迹的Python程序如下,请在方框中填写合适的代码,完善程序。
\columnratio{0.5}
\begin{paracol}{2}
\begin{lstlisting}
import matplotlib.pyplot as plt
def track(file):
    jd = []
    wd = []
    for line in open(file):
        line_data = line.split(',')
        `\clozeblank{2}`
        `\clozeblank{2}`
        jd.append(x)
        wd.append(y)    
    plt.plot(jd, wd)

filename = "Taxi_105.txt"
track(filename)
plt.show()
\end{lstlisting}
\switchcolumn
\includegraphics[width=\linewidth]{pic/c04.04.6taxi2}
\end{paracol}

\end{subquestions}

\begin{solution}
\begin{lstlisting}
x = float(line_data[2])
y = float(line_data[3])
\end{lstlisting}
\end{solution}


%% ============ 7
\setcounter{qnumber}{1}
\question ({\kaishu 作业本4.4})
小张扫描识别学生作业的客观题数据如图所示,编写Python程序分析各小题的答题情况,部分代码如下
\begin{figure}[!h]
\centering
\includegraphics[width=0.6\linewidth]{pic/c04.04.7answer1}
\end{figure}
\begin{subquestions}
\subquestion 读入数据后,若要输出数据的最后5行,请在划线处补充完程序。
\begin{lstlisting}
import pandas as pd
import matplotlib.pyplot as plt
df = pd.read_csv("hwdata.csv")
`\clozeblank{2}`
\end{lstlisting}

\subquestion 函数qcount(id)分析了题号是id的试题,学生各个选项的答题情况。如id=1,则统计ans1列中各个选项答题情况,上图中仅图示部分选A的有1人,选B的有8人。请完善程序。
\begin{lstlisting}
def qcount(id):
	col = "ans" + str(id)
    opt = df[`\clozeblank{1}`].value_counts()  # 根据列名称统计该列中不同值的个数
    opt = opt.sort_index()                 # opt对象中数据按索引升序排列
    print("第", id, "小题各选项的选答人数:")
    print(opt)                             # 输出各个选项的人数
    plt.figure(id)
    plt.pie(opt, labels=["A", "B", "C""D"]) # 绘制饼图
    plt.title("各选项的选答比例")
    plt.show()

\end{lstlisting}
若qnum保存了所有列标题,即\,\texttt{qnum=df.columns},则划线②处也可以填\key{qnum[id+1]}。

\subquestion 主程序。运行程序后输入数字“10”,结果如下图所示。请在划线③处补充完程序。该题错误率最高的选项是\key{~ D ~},占比\key{~ 19.63\% ~}。
\columnratio{0.53}
\begin{paracol}{2}
\begin{lstlisting}
sans ="BDCABDDBCB"  # 标准答案
while True:
    n = int(input("输人1-10的题号,
          查看答题情况;输入0结束:"))
    if `\clozeblank{2}`:
        qcount(n)  # 调用第(2)的函数
    elif n == 0:
        break
    else:
        print("输入错误,请重新输入!")

\end{lstlisting}
\switchcolumn
\includegraphics[width=\linewidth]{pic/c04.04.7answer2}
\end{paracol}

\end{subquestions}

\begin{solution}
\begin{lstlisting}
① df.tail() 或  df.tail(5)
(2)② col
qnum[id+1]
(3)③ 1 <= n <= 10
错误率最高D,占比19.63%
\end{lstlisting}
\end{solution}




\end{questions}
\end{groups}
