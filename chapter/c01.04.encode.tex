%\section{数据、信息与进位制}
\setcounter{section}{1}
\setcounter{subsection}{2}
\subsection{1.4 图像、音频编码、数据管理与安全}



\begin{compactenum}[1.]
%\labelwidth = 2em
\item \textbf{声音编码}

未压缩Wave格式文件的存储容量计算,本质是:采样点数 $\times$
每个点容量,具体地:采样频率(Hz)$\times$时间(s)$\times$声道数$\times$量化位数(单位:bit)。

\item \textbf{图像编码}

	\begin{compactitem}
	\item  数字图像包括矢量图形和位图图像。
	\item  矢量图形是用点、线或多边形等基于数学方程的几何图元表示的图像。如$x^2 + y^2 \le r^2$表示半径是$r$的实心圆。特点是:容量小、色彩少(8位色=256色)、放大不会失真
	\item  位图图像又称栅格图或点阵图,是由像素点(pixel)组成的。容量大、色彩丰富(32位真色彩可以到$2^{32}$色)、放大有锯齿状失真
	\end{compactitem}

\item \textbf{图像容量计算}

未压缩BMP格式文件的存储容量计算,本质上也是:像素点数 $\times$
每个点的容量,具体地:水平像素$\times$垂直像素$\times$颜色位深度(单位:bit)。已知颜色数量时,需要转成位深度,如256色就是8bit。

\item \textbf{数据管理}

	\begin{compactitem}
	\item  计算机数据管理经历了人工管理、文件管理和数据库管理三个阶段
	\item  数据库是数据管理的主要方式
	\item  现代数据库可以结构化数据、半结构化数据和非结构化数据
	\end{compactitem}

\item \textbf{文件格式}

	\begin{compactitem}
	\item  文档:\texttt{.txt}, \texttt{.docx}, \texttt{.wps}, \texttt{.html},  \texttt{.pdf}
	\item  图像:\texttt{.bmp}, \texttt{.jpg}, \texttt{.png}, \texttt{.gif}
	\item  音频:\texttt{.wav}, \texttt{.mp3}, \texttt{.wma}
	\item  视频:\texttt{.avi}, \texttt{.mpg}, \texttt{.mp4}, \texttt{.mov}
	\item  应用程序(可执行):\texttt{.exe}
	\end{compactitem}

\item \textbf{数据安全}

	\begin{compactitem}
	\item  威胁数据安全的因素:硬盘驱动器损坏、操作失误、黑客入侵、感染计算机病毒、遭受自然灾害等。
	\item  保护存储数据的介质(如硬盘)的安全:磁盘阵列、数据备份、异地容灾等手段。
	\item  提高数据本身的安全

	  \begin{compactitem}  
	  \item    数据加密:提高数据的\textbf{保密性}。通过加密算法和加密密钥将明文转变为密文,而解密则逆过程。
	  \item    数据校验:是为保证数据的\textbf{完整性}进行的一种验证操作。通常用一种指定的算法对原始数据计算出一个校验值,接收方按同样的算法计算出一个校验值,如果两次计算得到的校验值相同,则说明数据是完整的。常见的数据校验方法有MD5、CRC、SHA-1等。
	  \end{compactitem}
	\end{compactitem}


\end{compactenum}%


\begin{groups}


\group{单选题}{每题仅有一个正确选项}
\begin{questions}[rp]

%% =========== 1
\question
\smkai{(作业本1.3)}某Wave格式的音频文件,其采样频率为44.1kHz,量化位数为16bit,2个声道,其1秒的数据量为
\choice{176.4B}
{1.38KB}
{172.3KB}
{1.35MB}
\begin{solution}
C。计算公式:44100*16*2*1/8/1024=172KB
\end{solution}

%% =========== 2
\question
\smkai{(作业本1.3)}小明要参加学校多媒体作品比赛,其中对于图像类作品的要求:尺寸为$1280 \times 720$像素,颜色位深度为24位,其文件大小不得超过450KB。根据这一要求,小明上交的图像作品,其压缩比至少是
\choice{$2:1$}
{$4:1$}
{$6:1$}
{$8:1$}
\begin{solution}
C。计算公式:$\dfrac{1280 \times 720 \times 24}{8 \times 1024} : 450=6:1$
\end{solution}

%% =========== 3
\question
\smkai{(作业本1.3)}一部4K影片(3840×2160),其频为30fps(帧/秒),颜色位深度为24位,若不进行压缩,则其1分钟视频的数据量是
\choice{695MB}
{5.56GB}
{41.7GB}
{333.7GB}
\begin{solution}
C。计算公式:$\dfrac{3840 \times 2160 \times 24 \times 60 \times 30}{8 \times 1024 \times 1024 \times 1024}=41.7$
\end{solution}

%% =========== 4
\question
\smkai{(作业本1.4)}为了保障个人信息安全,下列措施有效的是
\choice{关闭防火墙软件}
{提升自身的信息安全意识}
{个人敏感信息保存在U盘中}
{个人账户的密码不要定期更改}
\begin{solution}
B。关闭防火墙增加了入侵风险;提升自身的信息安全意识,才能保障信息安全;个人敏感信息存放在U盘中,当U盘遗失或被他人查看到时可能会造成数据泄露;个人账户的密码应定期更改
\end{solution}

%% =========== 5
\question
\smkai{(作业本1.4)}2014年,中国铁路售票网站12306发生个人数据泄露事件,犯罪嫌疑人通过收集某游戏网站以及其他多个网站泄露的用户名与密码信息,尝试登录12306网站进行“撞库” 非法获取用户信息,谋取非法利益。结合该事例,为保护个人信息安全,下列行为最合理的是
\choice{用户要提高密码强度}
{不同网站采用不同的密码}
{每天改变一次密码}
{计算机安装杀毒软件}
\begin{solution}
B。本题A选项与D选项是能提高安全性的行为,但是本案例中,犯罪嫌疑人已经
通过其他途径获取了用户密码。在这个前提下,它们就不是本题的正确答案;

C选项表面上看提高了安全性,而实际上缺之可操作性毕竟人的记忆能力是有限的

本案例中,犯罪嫌疑人通过窃取用户在A网站中的账号信息,到B网站中批量登录,正是利用了人们为了方便记忆而对所有账户使用同一套账号和密码的行为习惯所实施的犯罪行为,所以在不同网站使用不同的密码是较为合理的,特别是某些非官方网站,可信度低,安全性弱,很容易泄露用户个人数据,所以答案是B
\end{solution}

%% =========== 6
\question
\smkai{(作业本1.4)}下列文件中属于网页文件的有
①\stt{hello.wav} ②\stt{index.htm} ③\stt{nothing.wma} ④\stt{admin.html} ⑤\stt{cookie.txt} ⑥\stt{bg.jpg}
\choice{①③}
{②④}
{②⑤}
{④⑥}
\begin{solution}
B。文件的类型一般可用文件扩展名区分。如`.htm`或`.html`为网页文件;`.wav`或`.wma`为音频文件;`.txt`为文本文件;`.jpg`为图像文件
\end{solution}

%% =========== 7
\question
\smkai{(作业本1.4)}对下列文件类型归类正确的是
\choice{\stt{movie.avi}视频,\stt{picnic.jpg}音乐}
{\stt{sky.mp3}音乐,\stt{psd.txt}文档}
{\stt{movie.avi}图片,\stt{brief.docx}文档}
{\stt{brief.docx}视频,\stt{game.rar}压缩}
\begin{solution}
B。`.avi`为视频文件,`.mp3`为音频文件,`.docx`为word文档,`.rar`为压缩文件
\end{solution}

%% =========== 8
\question
\smkai{(作业本1.4)}下列有关计算机数据管理的说法,\textbf{不正确}的是
\choice{不同的编码规则决定了不同的文件格式}
{计算机一般采用树形目录结构来管理文件}
{数据库系统的使用提高了数据的可维护性}
{传统数据库技术一般用于非结构化数据管理}
\begin{solution}
D。传统数据库技术只能处理结构化数据。不能处理非结构化数据
\end{solution}

%% =========== 9
\question
\smkai{(作业本1.4)}下列有关数据安全的说法。正确的是
\choice{经过加密后的数据就不会被破坏}
{通过数据校验可以检测数据完整性}
{提升数据安全,只需要做好存储介质的保扩}
{数据安全问题都由人为因素引发}
\begin{solution}
B。数据经过加密只是提高了保密性。还是会被破坏,数据校验常被用于检测数据完整性,提升数据安全不仅要做好存储介质的保护,还要提高数据本身的安全性,数据安全问题也可能由非人为因素引发,如火灾、地震等。
\end{solution}

%% =========== 10
\question
\smkai{(作业本1.4)}为了预防自然灾害引起的数据损坏,一般可采用的措施有
\choice{安装磁盘阵列系统}
{对数据进行加密}
{安装防火墙}
{建设异地容灾系统}
\begin{solution}
D。磁盘阵列、数据加密、防火墙都是对数据进行保护,在系统环境遭到破坏时。只有建设异地设施,才能保证系统正常运行
\end{solution}







\end{questions}
\end{groups}
